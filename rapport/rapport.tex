\documentclass{scrartcl}
\usepackage{cmap}
\usepackage[T1]{fontenc}
\usepackage{lmodern}
\usepackage[utf8]{inputenc}
\usepackage[french]{babel}

\title{Projet de Sémantique}
\author{%
  Jean \textsc{Caspar},
  Adrien \textsc{Mathieu}
}
\date{Dimanche 4 Juin 2023}

\begin{document}
\maketitle{}

\section*{Introduction}
Notre analyseur statique prend en charge les fonctions et le domaine relationnel des polyhèdre,
en plus de la base commune requise. Ces deux extensions sont compatibles, c'est-à-dire que les
contraintes ne sont pas locales à chaque fonctions mais sont, au contraire, prises en compte par
l'appelant (et, réciproquement, les contraintes pre-existant à un appel de fonction sont prises
en compte par l'appelé).\par
Tous les domaines pris en charge supportent un widening par paliers (sauf le domaine des
constantes, qui a un widening trivial). Les paliers en question sont
générés à partir de toutes les constantes trouvées dans le code source.

\section*{Appels de fonction}
Toutes les fonctions sont non-réentrantes, c'est-à-dire que si une fonction est appelée, elle ne
peut pas être appelée de nouveau avant de terminer.\par
L'analyse s'appuie sur cette propriété: quand un n\oe{}ud d'appel à une fonction \(f\) est
rencontré, la composante connexe de \(f\) est initialisé à \(\bot\), sauf le n\oe{}ud d'entrée
qui est initialisé avec l'environnement dans lequel se trouvait le n\oe{}ud d'appel à \(f\).
Puis, cette composante connexe est traitée entièrement. Enfin, l'environnement se trouvant dans
le dernier n\oe{}ud de la composante connexe est copié en sortie du n\oe{}ud d'appel à \(f\).\par
Ceci permet de conserver les contraintes ``à travers'' des appels de fonction.\par
Dans l'exemple \verb|poly/fonction1.c|, de l'information venant des contraintes avant un appel à
\verb|add| \emph{et} des contraintes venant de \verb|add| est utilisée par l'appelant.\par
Dans l'exemple \verb|poly/fonction2.c|, c'est l'inverse: l'appelé utilise des contraintes
obtenues par l'appelant.

\section*{Widening}
\subsection*{Comment s'effectue le widening}
Certains n\oe{}uds sont marqués comme nécessitant un widening. Pour ces n\oe{}uds \(n\), quand il faut
les mettre à jour, l'union des environnements des n\oe{}uds ayant une arrête entrante dans \(n\)
est calculée, et puis l'environnement de \(n\) est mis à jour par widening de l'ancienne valeur,
et de l'union calculée.\par
Aucune latence n'est introduite avant d'effectuer le widening.

\subsection*{Paliers}
Toutes les constantes trouvées dans le programme à analyser, plus ou moins un, sont utilisées
pour générer des paliers. Pour les domaines relationnels, toutes les contraintes de la forme
\(x\leq c\) et \(x\geq c\) sont utilisées comme paliers, pour toute variable \(x\) et toute
constante \(c\).\par
L'exemple \verb|constant_loop/constant_loop_1.c| exhibe un cas arbitrairement long à traiter
sans widening (en remplaçant `10` par un nombre arbitrairement grand), et où un widening sans
palier est trop brutal (on obtient une erreur de complétude).

\section*{Domaine relationnel}
L'implémentation des domaines relationnels est basé sur la librairie Apron. La partie que nous
avons écrite correspond à la ``traduction'' entre l'interface d'Apron et celle de notre
analyseur. Notons tout de même qu'Apron ne detecte pas très bien les environments bottom: si
une variable déclarée comme n'ayant que des valeurs entière ne peut avoir que des valeurs
non-entières (par exemple, \(x\in[1/2, 1/2]\)), Apron ne détectera pas qu'il s'agit d'un
environment bottom. Ainsi, ce cas est traité de façon ad-hoc dans notre code.

\section*{Notes d'implémentations}
Pour les domaines non-relationnels \(V\to\mathcal{P(V^\#)}\), où \(V\) est l'ensemble des
variables et \(\mathcal{V}^\#\) est le domaine abstrait des valeurs, si jamais il n'y a pas de
variables, il n'existe qu'une seule valeur dans ce domaine, on ne peut donc pas distinguer des
sections de code qui peuvent s'exécuter de celles qui ne le peuvent pas.\par
Avant correction, les exemples \verb|constant/constant_div_zero2.c|, \\
\verb|constant/constant_div_zero3.c| et \verb|constant/constant_div_zero4.c| exhibaient ce
problème (le premier reportait une failure alors qu'il ne devrait pas). Pour cette raison, une
variable est ajoutée au moment de l'initialisation si jamais aucun autre variable n'est présente.\par
De plus, une partie du code est dédié à la détection de divisions par zéro. C'est utile, d'une
part parce que Apron ne fait pas d'efforts pour ça (notamment, l'exemple
\verb|constant/constant_div_zero1.c| échoue si l'on ne fait rien, Apron considérant que \(x\) peut
avoir une valeur quelconque après assignation, alors qu'au contraire, aucune valeur n'est
possible!), d'autre part parce que cela permet de détecter une possible divergence d'un programme
même si la division par zéro ne fait pas intervenir de variables.

\end{document}
